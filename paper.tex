\documentclass[twocolumn,superscriptaddress,aps]{revtex4-1}

\usepackage[utf8]{inputenc}

\usepackage{amsfonts}
\usepackage{amssymb}
\usepackage{amsmath}
\usepackage{amsthm}

\usepackage{bm}
\usepackage{cancel}
\usepackage{bbold}
\usepackage{slashed}
\usepackage{graphicx}
\usepackage{color}
\usepackage{hyperref}
\usepackage{algorithm}
\usepackage{algpseudocode}
\usepackage{tikz}

\newcommand{\glnote}[1]{\textcolor{red}{[GL: #1]}}
\newcommand{\kcnote}[1]{\textcolor{red}{[KC: #1]}}

\theoremstyle{plain}
\newtheorem{theorem}{Theorem}
\newtheorem{proposition}[theorem]{Proposition}

\begin{document}


% ==============================================================================

\title{\Large{Adversarial Variational Optimization of Non-Differentiable Simulators}}
\vspace{1cm}
\author{\small{\bf Gilles Louppe}\thanks{\texttt{g.louppe@nyu.edu}}}
\affiliation{New York University}
\author{\small{\bf Kyle Cranmer}\thanks{\texttt{kyle.cranmer@nyu.edu}}}
\affiliation{New York University}

\begin{abstract}

In this note, ... \glnote{todo.}


\end{abstract}

\maketitle

% ==============================================================================

\section{Introduction}

\glnote{Prescribed vs. implicit. See case of non-diff models in Balaji et al.}


% ==============================================================================

\section{Problem statement}

We consider a family of parameterized densities $p_\mathbf{\theta}(\mathbf{x})$
defined implicitly through the simulation of a stochastic generative process,
where $\mathbf{x} \in \mathbb{R}^d$ is the data and $\mathbf{\theta}$ are the
parameters of interest. The simulation may involve some complicated latent
process, such that
\begin{equation}\label{eqn:p_x}
    p_\mathbf{\theta}(\mathbf{x}) = \int p_\mathbf{\theta}(\mathbf{x}|\mathbf{z}) p(\mathbf{z}) d\mathbf{z}
\end{equation}
where $\mathbf{z} \in \mathbb{R}^m$ is a latent variable providing an external source
of randomness.

We assume that we already have an accurate simulation of the stochastic
generative process that defines $p_\mathbf{\theta}(\mathbf{x}|\mathbf{z})$, as
specified through a deterministic function $g(\cdot; \theta) : \mathbb{R}^m \to
\mathbb{R}^d$. That is
\begin{equation}\label{eqn:p_x_sim}
    p_\mathbf{\theta}(\mathbf{x}) = \frac{\partial}{\partial x_1} \dots \frac{\partial}{\partial x_d} \int_{\{\mathbf{z}:g(\mathbf{z};\mathbf{\theta}) \leq \mathbf{x}\}} p(\mathbf{z}) d\mathbf{z}.
\end{equation}
The simulator $g$ is assumed to be a non-invertible function, that can only be
used to generate data in forward mode. For this reason, evaluating the integral
in Eqn.~\ref{eqn:p_x_sim} is intractable. Importantly, and as commonly found
in science, we finally assume the lack of access to or existence of derivatives of $g$ with respect to $\theta$, e.g. when specified as a computer program.

Given some observed data $\{ \mathbf{x}_i | i=1, \dots, N \}$ drawn from the
(unknown) true distribution $p_r$, our goal is the inference of the parameters
of interest $\mathbf{\theta}^*$ that minimize the divergence between $p_r$ and
the modeled data distribution $p_\mathbf{\theta}$ induced by $g(\cdot;
\mathbf{\theta})$ over $\mathbf{z}$. That is,
\begin{equation}
    \mathbf{\theta}^* = \arg \min_\mathbf{\theta} \rho(p_r, p_\mathbf{\theta}),
\end{equation}
where $\rho$ is some distance or divergence.


% ==============================================================================

\section{Background}

\subsection{Generative adversarial networks}

Generative adversarial networks (GANs) were first proposed by
\cite{goodfellow2014generative} as a way to build an implicit generative model
capable of producing samples from random noise $\mathbf{z}$. More specifically,
a generative model $g(\cdot; \mathbf{\theta})$ is pit against an adversarial
classifier $d(\cdot; \phi):\mathbb{R}^d \to [0,1]$ with parameters $\phi$ and whose antagonistic objective is to recognize real data $\mathbf{x}$
from generated data $g(\mathbf{z}; \mathbf{\theta})$. Both models $g$ and $d$
are trained simultaneously, in such a way that $g$ learns to maximally confuse
its adversary $d$ (which happens when $g$ produces samples comparable to the
observed data), while $d$ continuously adapts to changes in $g$. When $d$ is
trained to optimality before each parameter update of the generator, it can
be shown that the original adversarial learning procedure amounts to minimizing
the Jensen-Shannon divergence $\text{JSD}(p_r \parallel p_\theta)$ between $p_r$ and $p_\theta$.

As thoroughly explored in \citep{2017arXiv170104862A}, GANs remain remarkably
difficult to train because of vanishing gradients as $d$ saturates, or because of
unreliable updates when the training procedure is relaxed. As a remedy,
Wasserstein GANs~\citep{2017arXiv170107875A} reformulate the adversarial
setup in order to minimize the Wasserstein-1 distance $W(p_r, p_\theta)$ by
replacing the adversarial classifier with a 1-Lipschitz adversarial critic
$d(\cdot; \phi) : \mathbb{R}^d \to \mathbb{R}$. Under the WGAN-GP formulation of \cite{2017arXiv170400028G}
for stabilizing the optimization procedure,
training $d$ and $g$ results in alternating gradient updates on $\phi$ and $\theta$ in order to respectively minimize
\begin{align}
    {\cal L}_d =\,& \mathbb{E}_{\tilde{\mathbf{x}} \sim p_\theta} [d(\tilde{\mathbf{x}};\phi)] - \mathbb{E}_{\mathbf{x} \sim p_r} [d(\mathbf{x};\phi)]  \nonumber \\
                  & + \lambda \mathbb{E}_{\hat{\mathbf{x}} \sim p_{\hat{\mathbf{x}}}} [(|| \nabla_{\hat{\mathbf{x}}} d({\hat{\mathbf{x}}};\phi) ||_2 - 1)^2] \\
    {\cal L}_g =\,& -\mathbb{E}_{\tilde{\mathbf{x}} \sim p_\theta} [d(\tilde{\mathbf{x}};\phi)]
\end{align}
where ${\hat{\mathbf{x}}} := \epsilon \mathbf{x} + (1-\epsilon)\tilde{\mathbf{x}}$, for $\epsilon \sim U[0,1]$, $\mathbf{x} \sim p_r$ and $\tilde{\mathbf{x}} \sim p_\theta$.


\subsection{Variational optimization}

Following \citep{2012arXiv1212.4507S}, variational optimization (VO) (also known as the search gradient algorithm~\citep{2011arXiv1106.4487W}) is a general
optimization technique that can be used to form a differentiable bound
on the optima of a non-differentiable function. Given a function $f$ to minimize, VO
is based on the simple fact that
\begin{equation}
    \min_{\mathbf{c} \in {\cal C}} f(\mathbf{c}) \leq \mathbb{E}_{\mathbf{c} \sim q_\psi(\mathbf{c})} [f(\mathbf{c})] = U(\psi),
\end{equation}
where $q_\psi$ is a proposal distribution with parameters $\psi$ over input values $\mathbf{c}$.
That is, the minimum of a set of function values is always less than or equal
to any of their average. Provided that the proposal is flexible enough, the parameters $\psi$
can be updated to place its mass arbitrarily tight around the optimum $\mathbf{c}^* = \min_{\mathbf{c} \in {\cal C}} f(\mathbf{c})$.

Under mild restrictions outlined in  \citep{2012arXiv1212.4507S}, the bound $U(\psi)$ is differentiable, and using the log-likelihood trick it comes:
\begin{align}\label{eqn:vo-grad}
    \nabla_\psi U(\psi) &= \nabla_\psi \mathbb{E}_{\mathbf{c} \sim q_\psi(\mathbf{c})} [f(\mathbf{c})] \nonumber \\
    &= \nabla_\psi \int f(\mathbf{c})  q_\psi(\mathbf{c})  d\mathbf{c} \nonumber \\
    &= \int f(\mathbf{c}) \nabla_\psi q_\psi(\mathbf{c})  d\mathbf{c} \nonumber \\
    &= \int \left[ f(\mathbf{c}) \nabla_\psi \log q_\psi(\mathbf{c}) \right]  q_\psi(\mathbf{c})  d\mathbf{c} \nonumber \\
    &= \mathbb{E}_{\mathbf{c} \sim q_\psi(\mathbf{c})} [f(\mathbf{c}) \nabla_\psi \log q_\psi(\mathbf{c})]
\end{align}
Effectively, this means that provided that the score function $\nabla_\psi \log q_\psi(\mathbf{c})$ of the proposal
is known and that one can evaluate $f(\mathbf{\mathbf{c}})$ for any $\mathbf{c}$, then
one can construct empirical estimates of Eqn.~\ref{eqn:vo-grad}, which can
in turn be used to perform stochastic gradient descent (or a variant thereof)
in order to minimize $U(\psi)$.




% ==============================================================================

\section{Adversarial variational optimization}

\glnote{Naive approach by using VO on ${\cal L}_g$.}
\glnote{Smarter approach by exploiting the fact that $\partial d / \partial x$ is known exactly.}


% ==============================================================================

\section{Experiments}

\subsection{Toy problem}

\subsection{Physics example}


% ==============================================================================

\section{Related works}


\glnote{Implicit generative models.}
\glnote{ABC.}
\glnote{carl~\citep{cranmer2015approximating}.}
\glnote{Wood's papers.}
\glnote{CMA-ES.}


% ==============================================================================

\section{Summary}



% ==============================================================================

\section*{Acknowledgments}

GL and KL are both supported through NSF ACI-1450310, additionally KC is
supported through PHY-1505463 and PHY-1205376.


% ==============================================================================

\bibliographystyle{acm}
\bibliography{bibliography.bib}


\end{document}
