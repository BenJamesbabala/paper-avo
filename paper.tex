\documentclass[twocolumn,superscriptaddress,aps]{revtex4-1}

\usepackage[utf8]{inputenc}

\usepackage{amsfonts}
\usepackage{amssymb}
\usepackage{amsmath}
\usepackage{amsthm}

\usepackage{bm}
\usepackage{cancel}
\usepackage{bbold}
\usepackage{slashed}
\usepackage{graphicx}
\usepackage{color}
\usepackage{hyperref}
\usepackage{algorithm}
\usepackage{algpseudocode}
\usepackage{tikz}

\newcommand{\glnote}[1]{\textcolor{red}{[GL: #1]}}
\newcommand{\kcnote}[1]{\textcolor{red}{[KC: #1]}}

\theoremstyle{plain}
\newtheorem{theorem}{Theorem}
\newtheorem{proposition}[theorem]{Proposition}

\begin{document}


% ==============================================================================

\title{\Large{Adversarial Variational Optimization of Non-Differentiable Simulators}}
\vspace{1cm}
\author{\small{\bf Gilles Louppe}\thanks{\texttt{g.louppe@nyu.edu}}}
\affiliation{New York University}
\author{\small{\bf Kyle Cranmer}\thanks{\texttt{kyle.cranmer@nyu.edu}}}
\affiliation{New York University}

\begin{abstract}

Complex computer simulators are increasingly used across fields of science to
describe generative models tying parameters of an underlying theory to
experimental observations. Inference in this likelihood-free setup is often
intractable, as simulators rarely provide a way to evaluate the likehood
function for a given observation. In this note, we develop a likelihood-free
inference algorithm for fitting a forward non-differentiable generative model to
observed data. We adapt the adversarial training procedure of generative
adversarial networks by replacing the implicit generative network with a
domain-based scientific simulator, and solve the resulting non-differentiable
minimax problem with variational optimization. Effectively, the adversarial
variational optimization procedure results in learning an arbitrarily tight
proposal distribution over simulator parameters, such that the respective
hierarchical model matches the observed data.
\glnote{Mention experimental results.}
\glnote{Add 'so what?' conclusion.}


\end{abstract}

\maketitle

% ==============================================================================

\section{Introduction}

In many fields of science, computer simulators are used to describe complex
processes that tie parameters of an underlying theory to high dimensional
observations. In most cases, these implicit generative
models~\citep{2016arXiv161003483M} are specified as forward stochastic processes
that generate data. Because it is usually computationally intractable,
simulators often do not provide a way to directly evaluate the likelihood
function for a given observation,  which motivates a class of algorithms for
likelihood-free inference.

\glnote{State contributions clearly.}


% ==============================================================================

\section{Problem statement}
\label{sec:problem}

We consider a family of parameterized densities $p_\theta(\mathbf{x})$
defined implicitly through the simulation of a stochastic generative process,
where $\mathbf{x} \in \mathbb{R}^d$ is the data and $\theta$ are the
parameters of interest. The simulation may involve some complicated latent
process, such that
\begin{equation}\label{eqn:p_x}
    p_\theta(\mathbf{x}) = \int p_\theta(\mathbf{x}|\mathbf{z}) p(\mathbf{z}) d\mathbf{z}
\end{equation}
where $\mathbf{z} \in \mathbb{R}^m$ is a latent variable providing an external source
of randomness.

We assume that we already have an accurate simulation of the stochastic
generative process that defines $p_\theta(\mathbf{x}|\mathbf{z})$, as
specified through a deterministic function $g(\cdot; \theta) : \mathbb{R}^m \to
\mathbb{R}^d$. That is, we consider
\begin{equation}\label{eqn:p_theta}
    \mathbf{x} \sim p_\theta \equiv \mathbf{z} \sim p_z, \mathbf{x} = g(\mathbf{z}; \theta)
\end{equation}
such that the likelihood $p_\theta(\mathbf{x})$ can be rewritten as
\begin{equation}\label{eqn:p_x_sim}
    p_\theta(\mathbf{x}) = \frac{\partial}{\partial x_1} \dots \frac{\partial}{\partial x_d} \int_{\{\mathbf{z}:g(\mathbf{z};\theta) \leq \mathbf{x}\}} p(\mathbf{z}) d\mathbf{z}.
\end{equation}
Importantly, the simulator $g$ is assumed to be a non-invertible function, that can only be
used to generate data in forward mode. For this reason, evaluating the integral
in Eqn.~\ref{eqn:p_x_sim} is intractable. As commonly found
in science, we finally assume the lack of access to or existence of derivatives of $g$ with respect to $\theta$,
e.g. as when $g$ is specified as a computer program.

Given some observed data $\{ \mathbf{x}_i | i=1, \dots, N \}$ drawn from the
(unknown) true distribution $p_r$, our goal is the inference of the parameters
of interest $\theta^*$ that minimize the divergence between $p_r$ and
the modeled data distribution $p_\theta$ induced by $g(\cdot;
\theta)$ over $\mathbf{z}$. That is,
\begin{equation}
    \theta^* = \arg \min_\theta \rho(p_r, p_\theta),
\end{equation}
where $\rho$ is some distance or divergence.


% ==============================================================================

\section{Background}

\subsection{Generative adversarial networks}

Generative adversarial networks (GANs) were first proposed by
\cite{goodfellow2014generative} as a way to build an implicit generative model
capable of producing samples from random noise $\mathbf{z}$. More specifically,
a generative model $g(\cdot; \theta)$ is pit against an adversarial
classifier $d(\cdot; \phi):\mathbb{R}^d \to [0,1]$ with parameters $\phi$ and whose antagonistic objective is to recognize real data $\mathbf{x}$
from generated data $\tilde{\mathbf{x}} = g(\mathbf{z}; \theta)$. Both models $g$ and $d$
are trained simultaneously, in such a way that $g$ learns to maximally confuse
its adversary $d$ (which happens when $g$ produces samples comparable to the
observed data), while $d$ continuously adapts to changes in $g$. When $d$ is
trained to optimality before each parameter update of the generator, it can
be shown that the original adversarial learning procedure amounts to minimizing
the Jensen-Shannon divergence $\text{JSD}(p_r \parallel p_\theta)$ between $p_r$ and $p_\theta$.

As thoroughly explored in \citep{2017arXiv170104862A}, GANs remain remarkably
difficult to train because of vanishing gradients as $d$ saturates, or because of
unreliable updates when the training procedure is relaxed. As a remedy,
Wasserstein GANs~\citep{2017arXiv170107875A} reformulate the adversarial
setup in order to minimize the Wasserstein-1 distance $W(p_r, p_\theta)$ by
replacing the adversarial classifier with a 1-Lipschitz adversarial critic
$d(\cdot; \phi) : \mathbb{R}^d \to \mathbb{R}$. Under the WGAN-GP formulation of \cite{2017arXiv170400028G}
for stabilizing the optimization procedure,
training $d$ and $g$ results in alternating gradient updates on $\phi$ and $\theta$ in order to respectively minimize
\begin{align}
    {\cal L}_d =\,& \mathbb{E}_{\tilde{\mathbf{x}} \sim p_\theta} [d(\tilde{\mathbf{x}};\phi)] - \mathbb{E}_{\mathbf{x} \sim p_r} [d(\mathbf{x};\phi)]  \nonumber \\
                  & + \lambda \mathbb{E}_{\hat{\mathbf{x}} \sim p_{\hat{\mathbf{x}}}} [(|| \nabla_{\hat{\mathbf{x}}} d({\hat{\mathbf{x}}};\phi) ||_2 - 1)^2] \\
    {\cal L}_g =\,& -\mathbb{E}_{\tilde{\mathbf{x}} \sim p_\theta} [d(\tilde{\mathbf{x}};\phi)]
\end{align}
where ${\hat{\mathbf{x}}} := \epsilon \mathbf{x} + (1-\epsilon)\tilde{\mathbf{x}}$, for $\epsilon \sim U[0,1]$, $\mathbf{x} \sim p_r$ and $\tilde{\mathbf{x}} \sim p_\theta$.


\subsection{Variational optimization}

Following \citep{2012arXiv1212.4507S}, variational optimization (VO) (also known as the search gradient algorithm~\citep{2011arXiv1106.4487W} related to evolution strategies) is a general
optimization technique that can be used to form a differentiable bound
on the optima of a non-differentiable function. Given a function $f$ to minimize, VO
is based on the simple fact that
\begin{equation}
    \min_{\mathbf{c} \in {\cal C}} f(\mathbf{c}) \leq \mathbb{E}_{\mathbf{c} \sim q_\psi(\mathbf{c})} [f(\mathbf{c})] = U(\psi),
\end{equation}
where $q_\psi$ is a proposal distribution with parameters $\psi$ over input values $\mathbf{c}$.
That is, the minimum of a set of function values is always less than or equal
to any of their average. Provided that the proposal is flexible enough, the parameters $\psi$
can be updated to place its mass arbitrarily tight around the optimum $\mathbf{c}^* = \min_{\mathbf{c} \in {\cal C}} f(\mathbf{c})$.

Under mild restrictions outlined in  \citep{2012arXiv1212.4507S}, the bound $U(\psi)$ is differentiable, and using the log-likelihood trick it comes:
\begin{align}\label{eqn:vo-grad}
    \nabla_\psi U(\psi) &= \nabla_\psi \mathbb{E}_{\mathbf{c} \sim q_\psi(\mathbf{c})} [f(\mathbf{c})] \nonumber \\
    &= \nabla_\psi \int f(\mathbf{c})  q_\psi(\mathbf{c})  d\mathbf{c} \nonumber \\
    &= \int f(\mathbf{c}) \nabla_\psi q_\psi(\mathbf{c})  d\mathbf{c} \nonumber \\
    &= \int \left[ f(\mathbf{c}) \nabla_\psi \log q_\psi(\mathbf{c}) \right]  q_\psi(\mathbf{c})  d\mathbf{c} \nonumber \\
    &= \mathbb{E}_{\mathbf{c} \sim q_\psi(\mathbf{c})} [f(\mathbf{c}) \nabla_\psi \log q_\psi(\mathbf{c})]
\end{align}
Effectively, this means that provided that the score function $\nabla_\psi \log q_\psi(\mathbf{c})$ of the proposal
is known and that one can evaluate $f(\mathbf{\mathbf{c}})$ for any $\mathbf{c}$, then
one can construct empirical estimates of Eqn.~\ref{eqn:vo-grad}, which can
in turn be used to perform stochastic gradient descent (or a variant thereof)
in order to minimize $U(\psi)$.



% ==============================================================================

\section{Adversarial variational optimization}

The alternating stochastic gradient descent on ${\cal L}_d$ and ${\cal L}_g$ in
GANs inherently assume that the generator $g$ is a differentiable function. In
the setting where we are not interested in learning an implicit model but are
rather interested in the inference of parameters of a fixed
non-differentiable simulator (as outlined in Section~\ref{sec:problem}),
gradients $\nabla_\theta g$ either do not exist or cannot be accessed. As a
result, gradients $\nabla_\theta {\cal L}_g$ cannot be constructed and the
optimization procedure cannot be carried out.

In this work, we propose to perform variational optimization
on ${\cal L}_d$ and ${\cal L}_g$, thereby bypassing the non-differentiability
of $g$. More specifically, we consider a proposal distribution $q_\psi(\theta)$
over the parameters of $g$ and minimize in alternance the variational upper bounds
\begin{align}
    U_d &= \mathbb{E}_{\theta \sim q_\psi} [ {\cal L}_d ] \label{eqn:vo-ud} \\
    U_g &= \mathbb{E}_{\theta \sim q_\psi} [ {\cal L}_g ] \label{eqn:vo-ug}
\end{align}
respectively over $\phi$ and $\psi$.
When updating $d$, unbiased gradient estimates of $\nabla_\phi U_d$ can be
obtained by evaluating the exact and known gradient of $U_d$ over
mini-batches of true and generated data, as ordinarily
done in stochastic gradient descent. When updating $g$, gradient estimates
of $\nabla_\psi U_g$ can be derived with forward simulations, as described in Eqn.~\ref{eqn:vo-grad}.
That is,
\begin{equation}
    \nabla_\psi U_g = \mathbb{E}_{\theta \sim q_\psi(\theta), \mathbf{z} \sim p_z}  [-d(g(\mathbf{z};\theta);\phi) \nabla_\psi \log q_\psi(\theta)],
\end{equation}
which we can approximate with mini-batches of generated data
\begin{equation}
    \tilde{\nabla}_\psi U_g = \frac{1}{M} \sum_{m=1}^M -d(g(\mathbf{z}_m; \theta_m); \phi) \nabla_\psi \log q_\psi(\theta_m)
\end{equation}
for $\theta_m \sim q_\psi$ and $\mathbf{z}_m \sim p_z$.
\glnote{Can we exploit the fact that $\nabla_\mathbf{x} d(\mathbf{x})$ is known exactly for building better estimates $\tilde{\nabla}_\psi U_g$?}

Practically, the variational objectives \ref{eqn:vo-ud}-\ref{eqn:vo-ug}
have the effect of replacing the modeled data distribution of Eqn.~\ref{eqn:p_theta} with
a distribution parameterized in terms of $\psi$:
\begin{equation}\label{eqn:p_psi}
    \mathbf{x} \sim p_\psi \equiv \mathbf{z} \sim p_z, \theta \sim q_\psi, \mathbf{x} = g(\mathbf{z}; \theta).
\end{equation}
Intuitively, this corresponds to a family of simulators, each configured
with randomly sampled parameters $\theta \sim q_\psi$, whose joint collection
of generated samples is optimized with adversarial training to approach the real data distribution $p_r$.
However, this formulation does not necessarily guarantee that the
proposal distribution $q_\psi$ will place its mass arbitrarily tight
around the parameters of interest $\theta^*$. For this purpose, we augment Eqn.~\ref{eqn:vo-ug}
with a regularization term corresponding to the differential entropy $H$ of
the proposal distribution. That is,
\begin{equation}
    U_g = \mathbb{E}_{\theta \sim q_\psi} [ {\cal L}_g ] + \lambda H(q_\psi)
\end{equation}
where $\lambda \in \mathbb{R}^+$ is a hyperparameter controlling the trade-off
between the generator objective and the tightness of the proposal distribution.



% \glnote{
% While the above gradient estimate $\tilde{\nabla}_\psi U_g$ can readily be plugged
% into the adversarial training procedure, it may also exhibit a sampling
% variance that is larger than necessary, hence making the optimization unstable.
% However, one can exploit the fact that $-d(g(\mathbf{z}; \theta); \phi)$
% is a composition where $-d(\tilde{\mathbf{x}})$ is
% known and differentiable with respect to $\tilde{\mathbf{x}}$.
% In practice, using
% \begin{equation}
%     \tilde{\nabla}_\psi U_g = \frac{1}{M} \sum_{m=1}^M g(\mathbf{z}_m; \theta_m) \nabla_\psi \log q_\psi(\theta_m) \nabla_{\tilde{\mathbf{x}}} (-d(\tilde{\mathbf{x}}_m; \phi))
% \end{equation}
% works as good it seems, but I am not sure to precisely understand why, nor whether this is correct...
% It feels like this is approximating the chain rule $\nabla_\theta g(\theta)  \nabla_{\tilde{\mathbf{x}}} (-d(\tilde{\mathbf{x}}))$.
% With large mini-batches, both approaches seem to work fine, though the former seems to be less stable.
% }









% ==============================================================================

\section{Experiments}

\subsection{Toy problem}

\subsection{Physics example}


% ==============================================================================

\section{Related works}


\glnote{Implicit generative models.}
\glnote{classifier ABC.}
\glnote{carl~\citep{cranmer2015approximating}.}
\glnote{CMA-ES.}


% ==============================================================================

\section{Summary}



% ==============================================================================

\section*{Acknowledgments}

GL and KL are both supported through NSF ACI-1450310, additionally KC is
supported through PHY-1505463 and PHY-1205376.


% ==============================================================================

\bibliographystyle{acm}
\bibliography{bibliography.bib}


\end{document}
